% preamble
\documentclass{article}
\usepackage{fullpage}
\usepackage{fancyhdr}
\setlength{\headheight}{12pt}
\usepackage[headsep=1cm,total={6.5in, 8.5in}]{geometry}

% useful packages
\usepackage{amsmath,amssymb, amsthm}

\usepackage[shortlabels]{enumitem}

% add your own packages here
\usepackage{verbatim}

% commands for header, fill in the appropriate values
\newcommand{\codename}	{Snoop Matroidd} % type name here
\newcommand{\settitle}	{Problem Set 2} %type set title here (ex. Problem Set 3, Midterm)
\newcommand{\latesub}	{No} % type yes/no indicating late submission

% a few helpful commands 
\newcommand{\RR}{\mathbb{R}} % typing \RR prints the Blackboard R used for Real Numbers
\newcommand{\NN}{\mathbb{N}} % typing \NN prints the Blackboard N used for Natural Numbers
\newcommand{\ZZ}{\mathbb{Z}} % typing \ZZ prints the Blackboard Z used for Integers

% construct your own commands here

\newcommand{\GCD}{\textrm{GCD}}


% commands for header, don't adjust
\begin{document}
\thispagestyle{empty}
\begin{center}\framebox{\vbox{ \vspace{2mm}
                  \hbox to 6.5in {\textbf{CS~38~~~Algorithms}\hfill \textbf{\today}} \vspace{4mm}
                  \hbox to 6.5in {\Large \hfill \settitle{}  \hfill} \vspace{2mm}
                  \hbox to 6.5in {\textbf{Codename: } \codename{} \hfill \textbf{Late Submission: } \latesub} \vspace{2mm}}
      }
\end{center} \vspace*{1mm}
\pagestyle{fancy}\lhead{Codename: \codename} \chead{\settitle} \rhead{\today}

% insert solution here
\section{Problem 1 (6 points)}
Collaborated with: No one.

\paragraph{\indent}
In the following recurrence problems, you may suppose: (a) For \(1 \leq n \leq n_0\), \(T(n) = 1\) and \(f(n) = 1\);
(b) Rounding the arguments of \(T\) or \(f\) has no effect on the asymptotics. One way to deal with this
is simply to consider \(T\) and \(f\) extended to all real \(n \geq 1\), where for non-integer n we define \(f(n)\) to
be \(f(\lfloor n \rfloor)\).

\begin{enumerate}[(a)]
      \item If \(\exists c > 1, n_0\), s.t. \(\forall n > n_0, \sum_i f(n/b_i) \geq
            cf(n)\), then \(T(n) \in \Theta(\#leaves) = \Theta(n^w) \).
            \newline(The runtime is dominated by thr work at the leaves of the recursion tree.)
      \item If \(f(n) \in \Theta(n^w)\), then \(T(n) \in \Theta(n^w \log n)\).
            \newline(The work done at each level is approximately equal.)
      \item If \(\exists c<1, n_0\), s.t. \(\forall n > n_0, \sum_i f(n/b_i) \leq cf(n)\),
            then \(T(n) \in \Theta(f(n))\).
            \newline(Most of the work is done at the root.)
\end{enumerate}

\subsubsection{Proposition}
\paragraph{\indent}
Consider the runtime recurrence \(T(1) = 1\), \(T(n) = n \log n + 2T(\lfloor n/2 \rfloor)\). Explain why
the generalized master theorem does not solve this recurrence for us. Solve it
anyway. Explain why the generalized master theorem given above does not solve this
recurrence for us (you do not need to prove anything about this theorem). Solve it
anyway. (Precisely, find an explicit function \(f(n)\) such that you can show that there exists
constants \(0 < c < C\) such that \(c f(n) \leq T(n) \leq C f(n)\) for all large enough \(n\). Partial
credit if you can only get one of the two inequalities.)
\subsubsection{Proof}
\begin{proof} Proof that the GMT doesn't hold for this function.
      \begin{enumerate}
            \item The runtime recurrence given needs to be solved with the substitution method as
                  the generalized master theorem does not give us a recurrence relation
                  that we can solve.
            \item We will prove the prior assertion by assuming we can use any of cases
                  (a), (b), and (c), and then then finding a contradiction in our proof.
            \item  \begin{proof} First case of the GMT
                        \begin{enumerate}[(i)]
                              \item Assume that case (a) holds, s.t. \(b_1 = 2\) and that \(b_2 = 2\).
                              \item Observe that \(\sum_i f(n/b_i) = 2f(n/2)\).
                              \item Observe that, for some value of \(c>1\), \(2f(n/2) \geq cf(n)\).
                              \item The given function is defined as \(f(n) = n \log n \).
                              \item \(\therefore \exists c > 1 \implies n \log (n/2) \geq c n \log n \).
                              \item Contradiction! \(\log (n/2) < \log (n)\).
                              \item \( \therefore \nexists c > 1 \).
                              \item \( \therefore \) case (a) doesn't hold.
                        \end{enumerate}
                  \end{proof}
            \item  \begin{proof} Second case of the GMT
                        \begin{enumerate}[(i)]
                              \item Assume case (b) holds, s.t. \(w=1\).
                              \item \(\implies n \log n  \in \Theta (n)\).
                              \item Contradiction! This violates the definition of Big
                                    Theta notation.
                              \item \(\therefore \) case (b) doesn't hold.
                        \end{enumerate}
                  \end{proof}
            \item \begin{proof} Third case of the GMT
                        \begin{enumerate}[(i)]
                              \item Assume case (c) holds, s.t. \(b_1 = b_2 = 2\).
                              \item \( \implies \exists c < 1; \sum_i f(n/ b_i) = 2f(n/2) \leq cf(n)\).
                              \item \( \implies n \log (n/2) \leq  cn \log n \).
                              \item \( \implies \log (n) - \log (2) \leq c \log n\).
                              \item \( \implies \log 2 \geq \log (n) * (1-c)\).
                              \item Contradiction! \( \nexists c < 1\) that holds for
                                    every possible value of \(n\).
                              \item \( \therefore \) Case (c) doesn't hold.
                        \end{enumerate}
                  \end{proof}
            \item Having proved that each case of the Generalized Master Theory fails to
                  hold for our given function, we have proved that the GMT cannot solve
                  this recurrence relation.
      \end{enumerate}
\end{proof}
\subsubsection{Solution}
\begin{enumerate}
      \item To solve for the runtime recurrence \(T(n)\) we can use the substitution
            method from the TA notes.
      \item Without loss of generality, assume that \(1 < \log (n_0)\)
      \item Define \( c_a = MIN(k_a, (1+\log ^2(2)) /( 2 \log 2))\), s.t. \(k_a > 0 \implies k_a n
            \log^2(n) \leq T(n) \forall n \leq n_0 \).
      \item Define \( c_b = MAX(k_b, (1 + \log^2 (2)) / (2 \log 2))\), s.t. \(k_b n \log^2
            (n) \geq T(n) \forall n \leq n_0\).
      \item \begin{proof} Use induction to prove that \(\forall n, c_a n \log^2 (n) \leq T(n) \leq c_a n
                  \log^2(n)\).
                  \begin{enumerate}
                        \item \begin{proof}
                                    Base Case: \(n \leq n_0\)
                                    \begin{enumerate}
                                          \item Observe that \(c_a n \log^2 (n) \leq k_a n
                                                \log^2 (n) \leq T(n) \leq k_b n \log^2(n)
                                                \leq c_b n \log^2 (n)\).
                                          \item This follows our assumption so far.
                                    \end{enumerate}
                              \end{proof}
                        \item Assume \(\forall j; j < n\), \(c_a j \log^2(n) \leq
                              T(j) \leq c_b j \log^2 (j) \)
                        \item \begin{proof}
                                    Inductive case for the lower bound
                                    \begin{enumerate}
                                          \item \(T(n) = n \log n + 2 T(n/2)\)
                                          \item \( \implies T(n) \leq n \log n  + c_a n \log^2 (n/2)\).
                                          \item \( \implies T(n) \leq n \log n + c_a n ( \log^2 (n) -
                                                2 \log (2) \log(n) + \log^2(2)) \).
                                          \item \( \implies T(n) \leq c_b n \log^2 (n) + n \log (n)
                                                (1-2 c_b \log(2)) + n \log^2 (2)\).
                                          \item \((1+\log^2(2)/(2\log (2))) \geq c_a\).
                                          \item \( \implies 0 \geq 1 - 2 c_a \log (2) + \log^2 (2)\).
                                          \item \( 1 \leq \log (n_0) \leq \log (n)
                                                \implies c_a n \log^2 n \geq T(n)\).
                                    \end{enumerate}
                              \end{proof}
                        \item \begin{proof}
                                    Inductive case for upper bound
                                    \begin{enumerate}
                                          \item \(T(n) = n \log n + 2 T(n/2)\)
                                          \item \( \implies T(n) \leq n \log n  + c_b n \log^2 (n/2)\).
                                          \item \( \implies T(n) \leq n \log n + c_b n ( \log^2 (n) -
                                                2 \log (2) \log(n) + \log^2(2)) \).
                                          \item \( \implies T(n) \leq c_b n \log^2 (n) + n \log (n)
                                                (1-2 c_b \log(2)) + n \log^2 (2)\).
                                          \item \((1+\log^2(2)/(2\log (2))) \leq c_b\).
                                          \item \( \implies 0 \geq 1 - 2 c_b \log (2) + \log^2 (2)\).
                                          \item \( 1 \leq \log (n_0) \leq \log (n)
                                                \implies c_b n \log^2 n \geq T(n)\).
                                    \end{enumerate}
                              \end{proof}
                        \item \( \therefore \forall n, c_a n \log^2(n) \leq T(n) \leq c_b
                              n \log^2(n) \implies T(n) \in \Theta(n \log^2 n)\).
                  \end{enumerate}
            \end{proof}
\end{enumerate}

\newpage
\section{Problem 2 (6 points) \textit{Quickersort}}
Collaborated with: No one.

\paragraph{\indent}
As we saw, the randomized version of Quicksort is a wonderfully efficient and
practical sorting algorithm. Indeed the expected number of comparisons it performs
is within a small constant factor of best possible (although we've not yet seen the proof of the
lower bound on the complexity of randomized sorting algorithms).
\paragraph*{\indent}
Nonetheless, in this exercise we'll see an algorithm that is even a bit better.
\textit{Quickersort} works like this: Pick \textit{three} out of the n inputs, uniformly but without repetition.
Determine the median of the three elements, then use it as the pivot. I.e., compare all elements
against the pivot, then recursively run Quickersort on the two resulting sets of inputs.
Performing an exact time analysis of the algorithm is a bit complicated but a “continuum
limit” is not too bad and that's what you are to do in this exercise.
\subsection{Part (a)}
\subsubsection{Proposition}

\paragraph{\indent}
Letting \(xn\) denote the rank of the pivot, show that in the limit of large \(n\),
\(x\) is distributed according to the probability density \(6x(1 - x)\) on \([0, 1]\).
That is to say, for all \(0 \leq t \leq 1\),
\(\Pr[x < t] \rightarrow \int_0^t  6y(1-y) \, \mathrm{d}y\).

\subsubsection{Proof}
\begin{proof}
      \begin{enumerate}
            \item In the case of a pool with size of large \(n\), the non-independence of the 3 draws we
                  do is not going to be relevant, and we can consider each draw as
                  independent.
            \item The chance of drawing an entry with a rank of \(xn\) is uniformly
                  distributed from (large) \(n\) elements in the pool, and as such is \(~
                  1/n\).
            \item Under our limiting assumption, 1/n goes to 0, and the chance of drawing
                  an element with a smaller rank than \(xn\) is \((xn -1)/n\), so we can approximate
                  the chance as \(x\).
            \item The chance of drawing an element with a higher rank than \(xn\) is
                  \((1-x)\), as there are \(1-x\) elements with a larger rank.
            \item We have \(3! \) ways to choose the three draws.
            \item \( \therefore \Pr[x] = 6 * (1/n) * x * (1-x) \), which reduces to \(
                  (6x/n)(1-x) \).
            \item Now we sum the probabilities over the range \([0,1]\) to find the
                  probability that \(x < t\), and take the limit over large \(n\) to pull
                  back out the integral \( \int_0^t{6x^{\prime}(1-x^{\prime})
                  dx^{\prime}}\).
            \item And thus the probability density function is \(6x(1-x)\) over \([0,1]\).

      \end{enumerate}
\end{proof}

\subsection{Part (b)}
\subsubsection{Proposition}
\paragraph{\indent}
Let \(T(n)\) be the expected number of comparisons performed
by Quickersort on a worst-case input of size n. Explain why \(T(n)\) satisfies the recurrence
\[T(n) \leq D*n + 2 \int_0^1  6x(1-x) T(xn) \, \mathrm{d}x, \]
(I redefined the variable \(d\) in the proposition given to \(D\) for my own internal
order) and I fix the value of \(D\) because the continuum limit allows me to drop off
terms in the recurrence that are less than linear in \(n\).
\subsubsection{Proof}
\begin{proof}
      \begin{enumerate}
            \item \(T(xn)\) takes the place of \(x\) in the expected value of the probility density for
                  the cases up to \(xn\)  \[\int_0^1  6x(1-x) T(xn) \, \mathrm{d}x.\]
            \item \(T((1-x)n)\)  takes the place of \(x\) in the expected value of the probility density
                  for the cases greater than \(xn\)  \[\int_0^1  6x(1-x) T((1-x)n) \, \mathrm{d}x\]
            \item We can add these two integral cases together by recognizing they were a single integral
                  that integrates \(\int_0^1  6x(1-x) (T((1-x)n)T(xn)) \, \mathrm{d}x\) as we have a worst
                  case that needs to sort both arrays around the pivot \(xn\) and \((1-x)n\)
            \item So we can represent them as
                  \[2\int_0^1  6x(1-x) T(xn) \, \mathrm{d}x\]
            \item \(D*n\) is the minimum time it takes to compares each element to the pivot, and since
                  we treat comparison as an \(O(1)\) operation, and we have \(n\) elements, we find
                  \(D*n=n\) and \(D=1\).
      \end{enumerate}
\end{proof}

\subsection{Part (c)}
\subsubsection{Proposition}

\paragraph{\indent}
Guess that \(T(n)\) is upper bounded by a function of the form \(nb(c + \log n)\) (here log is
natural log). Solve the recurrence and show that it can be satisfied by taking the leading
constant \(b = 12/7\)  improving on Quicksort's leading constant of 2. You may find the
following definite integrals useful: \(\int_0^1 \! x^{2}(1-x) \, \mathrm{d}x = 1/12\)
\(\int_0^1 \! x^{2}(1-x)\log x \, \mathrm{d}x = -7/144\)
\subsubsection{Proof}
\begin{proof} We will use the substitution method from the TA notes.
      \begin{enumerate}

            \item \begin{proof}
                        Base Case n:
                        \begin{enumerate}
                              \item If we choose that \(c\) is picked such that \(n\) is
                                    below our continuum limit assumption, than the upper
                                    bound we set out to prove holds for the base case
                                    \(n\) for such a \(c\).
                        \end{enumerate}
                  \end{proof}
            \item \begin{proof}

                        Inductive Case \(n < N\):

                        \begin{enumerate}
                              \item Let \(T(xn) = xn*b(c+\log (x n))\) and \(T(n) \leq (1)*n
                                    + 12*\int_0^1 ! x^2(1-x)(nb(c+\log (xn)))\, \mathrm{d}x\)
                              \item \( n + 12nb \int_0^1 (1-x) x^2 (c + \log (xn) ) \, dx\)
                              \item \( n + 12nb \int_0^1 c(1-x) x^2  + (1-x) x^2 \log (xn)  \, dx\)
                              \item \( n + 12nb (\int_0^1 c(1-x) x^2 \, dx + \int_0^1 (1-x) x^2 \log (xn) \, dx)\)
                              \item \( n + 12nb (c \int_0^1 (1-x) x^2 \, dx + \int_0^1 (1-x) x^2 (\log(x) + \log(n)) \, dx)\)
                              \item \( n + 12nb (c \int_0^1 (1-x) x^2 \, dx + \int_0^1 (1-x) x^2 \log(x) + (1-x) x^2 \log(n) \, dx)\)
                              \item \( n + 12nb (c \int_0^1 (1-x) x^2 \, dx + \int_0^1 (1-x) x^2 \log(x) \, dx + \int_0^1 (1-x) x^2 \log(n) \, dx)\)
                              \item \( n + 12nb (c \int_0^1 (1-x) x^2 \, dx + \int_0^1 (1-x) x^2 \log(x) \, dx + \log(n)\int_0^1 (1-x) x^2 \, dx)\)
                              \item Now replace with the given helpful definite intergrals.
                              \item \( n + 12nb (c (1/12) + (-7/144) + \log(n)(1/12))\)
                              \item \( n + nbc - nb(7/12) + nb \log(n)\)
                              \item \( n + \frac{1}{12} b n (12 c+12 \log (n)-7)\)
                              \item If we sub in \(b=12/7\) we get \( n + \frac{1}{12} b n (12 c+12 \log (n)-7)\)
                              \item \(T(N) \leq n  + (144/7)( (1/12) nc - (7/144)n
                                    +(1/12)n \log (n)         )\)
                              \item \(T(N) \leq (12/7) n (c+ \log (n))\)

                              \item \( T(n) \leq 12/7 n (c + Log[n])\)

                        \end{enumerate}

                  \end{proof}

            \item Thus we see that using \(b=12/7\) provides a more favorable leading constant that improves
                  upon Quicksort's.
      \end{enumerate}
\end{proof}



\newpage
\section{Problem 3 (6 points)}
Collaborated with: No one

\subsection{Proposition}
Let \(\Sigma \) be a constant-size alphabet. A string \(y = (y_1,\ldots, y_n)\)
\(y_i \in \Sigma \)  has \(x = (x_1,\dots, x_m)\) as a subsequence if there
are indices \(a_1 < \cdots < a_m\) such that \(x_j = y_{a_j}\)
for all \(1 \leq j \leq m\)
A palindrome over \(\Sigma \) is a string \(x = (x_1,\dots, x_m), m \geq 0\)  such that
\(x = (x_m,\dots, x_1)\)
Devise an algorithm that on input \(y = (y_1,\dots, y_n)\) returns the length of its longest
subsequence that is a palindrome. For full credit your algorithm should run in time \(O(n^2)\)

\subsection{Algorithm}
\paragraph{\indent}We will use DP to populate a table such that a specific
final entry in the table contains the length of the longest subsequence that is a palindrome.

\paragraph{\indent}Construct a square two dimensional table for DP \(LSP[i,j]\)
with \(n^2\) entries and dimensions of \(n \times n\)  and initialize each entry with \(0\)
We then initialize the diagonal \(LSP[i,i]\) to 1 to represent the palindrome of length 1 at each
point.
To populate the table we are going to loop \(n^2\) times iterating \(i\) and \(j\) from
\(i=1\) through \(i=n\)  and from \(j=1\) to \(j=n\)
We set the values of \(LSP[i,j]\) according to the following if statements, applied in the order given:
\begin{enumerate}
      \item If \(i=j\)  then set \(LSP[i,j] \leftarrow 1\)
      \item Else-If \((i < j)\) AND \((y_i == y_j)\)  then set \(LSP[i,j] \leftarrow LSP[i+1,j-1]+2\)
      \item Else-If \((i < j)\) AND \((y_i \neq y_j)\)  then set
            \(LSP[i,j] \leftarrow MAX(LSP[i+1,j],LSP[i,j-1])\)
      \item Else-If \((i > j)\)  then set \(LSP[i,j] \leftarrow 0\)
\end{enumerate}
After we complete the \(n^2\) iterations through the loops we query the value at \(LSP[1,n]\) for
the length of the Longest Subsequence Palindrome of our input string \(y\)
\begin{proof}[Proof of Correctness]
      \begin{enumerate}
            \item We define a palindrome over our alphabet \(\Sigma \) as a string
                  \(x = (x_1,\dots, x_m), m \geq 0\)  such that \(x = (x_m,\dots, x_1)\)
            \item I assert that the number at \(LSP[i,j]\) is the length of the longest palindrome in
                  \((y_i,\dots,y_j)\in y\)
            \item If \(LSP[i,j] == 0\)  then the entry has only been initialized.
            \item If our (sub) string has a length of \(1\)  it is a palindrome of the same length,
                  and so we can conlude that \(LSP[i,j]\) should equal \(1\) if \(i == j\) (our 1st rule).
            \item Define a helper function \(L\) that takes a string \(y\) and two indexes \(i,j\)
                  and recursively applies the aforementioned rules to populate our DP table by
                  finding the LSP \(\in y\)
            \item \(i\) must be less than \(j\)  as \(i\) is the start of a subsequence and \(j\) the end.
                  We don't allow negative lengths, so if we are given such an arrangement set
                  \(LSP[i,j] \leftarrow 0\)  and we can show negative lengths never enter by proving our rules.
            \item Assume \(i < j\)  the length of \((y_i,\dots,y_j)\) must be \(\geq 2\)
            \item
                  \begin{proof}[1st Lemma: Rule 2]
                        \begin{enumerate}
                              \item We will use induction to prove our second rule is
                                    correct.
                              \item Base Cases: \begin{enumerate}
                                          \item If we have a string of length 2 as ''11'', then \(LSP[1,2]\) should
                                                be equal to \(LSP[2,1]+2\) which is \(0+2=2\)
                                          \item If we have a string of length 3 as ''121'', then \(LSP[1,3]\)
                                                should be equal to \(LSP[2,2] + 2\) which is \(1+2=3\)
                                          \item \(LSP[2,1]\) may not be \(<0\)  otherwise we have a contradiction.
                                    \end{enumerate}
                              \item Induction Cases: \begin{enumerate}
                                          \item We begin with assuming that this algorithm
                                                is valid for our base cases with small \(n\)
                                                for some \(y = (y_1,\dots,y_n)\)  and that the LSP of
                                                \(y\) is \(p = (p_1,\dots,p_m)\)
                                          \item If we insert some character \(c \in \Sigma
                                                \) at both the start
                                                and the end of \(y\)
                                                then we have constructed a new string \(y^{\prime} = (y^{\prime}_1,\dots,y^{\prime}_{n+2})\)
                                          \item If we insert the same \(c\) at the beginning and end of \(p\) we also now have
                                                \(p^{\prime} = (p^{\prime}_1,\dots,p^{\prime}_{m+2})\)
                                          \item Observe that \(y = (y^{\prime}_2,\dots,y^{\prime}_{n+1})\) and \(p = (p^{\prime}_2,\dots,p^{\prime}_{k+1})\)
                                          \item Because \(p\) is already a palindrome, we can see that \(p'\) has the property
                                                that \(p^{\prime}_i = p^{\prime}_{(m+2)+1-i}\) where \((2 \leq i \leq m+1)\)  and that since
                                                \(p^{\prime}_1 = p^{\prime}_{m+2}\) we can see that \(p^{\prime}_i =p^{\prime}_{(m+2)+1-i}\) where
                                                \((1 \leq i \leq m+2)\)
                                          \item Thus we see that \(p^{\prime}\) is also a palindrome.
                                          \item Now we need to check if \(p^{\prime}\) is the longest subsequence
                                                palindrome of \(y^{\prime}\)
                                                Start by assuming that it is not the longest.
                                          \item If \(p^{\prime}\) is not the longest subsequence palindrome, then we can find some
                                                palindrome \(p^{\prime\prime} \in y^{\prime}\) that has a bigger length than \(p^{\prime}\)  which necessitates
                                                that \(p^{\prime\prime}\) is at least 3 characters longer than \(p\)
                                          \item If there is a \(c\) at the beginning and end of \(p^{\prime\prime}\)  then \(p^{\prime\prime}\) can remove
                                                the two \(c\) characters at the beginning and end and find a palindrome \(\in y\)
                                                that has a length greater than \(p\)  Contradiction!: \(p\) was the LSP of \(y\)
                                          \item If there is no \(c\) at either the beginning or the end of \(p^{\prime\prime}\)  and \(p^{\prime\prime}\)
                                                is \(\in y\)  \(p^{\prime\prime}\) must be the same length as \(p\)  and \(p^{\prime\prime}\) has a length
                                                less than that of \(p^{\prime}\)  Contradiction!: \(p^{\prime\prime}\) is supposed to be an LSP \(\in y^{\prime}\)
                                          \item If there is only one \(c\) at either the beginning or the end, then
                                                if \(p^{\prime\prime}\) has \(c = y^{\prime}_1 = p''_1\) but doesn't have \(c = y^{\prime}_{n+2} = p^{\prime\prime}_{m+3}\)
                                                then the length of \(p^{\prime\prime}\) must be 1 if there is no \(c \in (y'_2,\dots,y'_{n+1})\)
                                                If there is a \(c \in (y^{\prime}_2,\dots,y^{\prime}_{n+1})\) we can label the index of some \(c\)
                                                as \(k, y^{\prime}_k\)  There must exist a palindrome \(p^{\prime\prime\prime} \in (y^{\prime}_1,\dots,y^{\prime}_k)\)  and
                                                \(p'' = y^{\prime}_1 + p^{\prime\prime\prime} + y^{\prime}_k\)  and \(p\prime\prime\prime \in (y'_2,\dots,y^{\prime}_{n+1})\)  We assumed
                                                that \(p\) is an LSP \(\in
                                                (y^{\prime}_2,\dots,y^{\prime}_{n+1})\)  thus \(p\prime\prime\prime \)
                                                isn't longer than
                                                \(p\)  and thus the length \(|p''| = |p\prime\prime\prime| + 2\) is less than or equal to
                                                \(|p^{\prime}| = |p| + 2\)  If the tail end element of \(p^{\prime\prime}\) is \(c\)  then this plays
                                                out the same for \(p^{\prime\prime}_1 =
                                                c\) by symmetry, and observe that
                                                \(\nexists p^{\prime\prime}\)
                                                where
                                                \(|p^{\prime\prime}| > |p^{\prime}|\)
                                          \item Thus we can see that if \(p\) is an LSP \(p \in y\)  then
                                                \(p^{\prime}\) is an LSP \(p^{\prime} \in y'\)
                                          \item By induction we can see that if the application of this rule holds for
                                                strings of length \(n\)  then it should also hold for strings of length \(n+2\)
                                    \end{enumerate}
                        \end{enumerate}
                  \end{proof}

            \item
                  \begin{proof}[2nd Lemma: Rule 3]
                        \begin{enumerate}
                              \item We want to prove that if \((i < j)\) AND \((y_i \neq y_j)\) then the length \(|y'|\)
                                    of the LSP \(\in y' = (y_i,\dots,y_j)\) is equal to \(|y'| =
                                    MAX(L((y_{i+1},\dots,y_j),i+1,j),L((y_i,\dots,y_{j-1}),i,j-1))\).
                              \item Base Cases:
                                    \begin{enumerate}
                                          \item If we have some \(y = ('1', '2')\) and \(|y|=2\)  then we can see that
                                                \(LSP[1,2] = MAX(LSP[1,1],LSP[2,2]) = MAX(1,1) = 1\).
                                          \item if \(|y|=3\) and \(y = ('1','1','2')\)  then we can see that \(L(('1','1'),1,2)=2\)
                                                and \(L(('1','2'),2,3)=1\)  thus the LSP of \(y\) is 2.
                                    \end{enumerate}
                              \item Induction Cases:
                                    \begin{enumerate}
                                          \item Assume we have a string \(y = (y_1,\dots,y_n)\) where \(y_1 \neq y_n\).
                                          \item Assume our algorithm is valid \(\forall y\) where \(|y| \leq n-1, n \geq 2\).
                                          \item Assume \(p\) is an LSP \(p \in y = (y_1,\dots,y_{n-1})\).
                                          \item Assume \(p'\) is an LSP \(p' \in y' = (y_2,\dots,y_n)\).
                                          \item Assume \(p''\) is an LSP \(p'' \in y = (y_1,\dots,y_n)\).
                                          \item Thus, since \(y_1 \neq y_n\)  then \(p''\) must not have \(y_1\) as its head
                                                and \(y_n\) as its tail. So \(p'' \in (y_1,\dots,y_{n-1})\) OR \((y_2,\dots,y_n)\).
                                          \item If \(p'' \in (y_1,\dots,y_{n-1})\)  then \(p'' == p\).
                                          \item If \(p'' \notin (y_1,\dots,y_{n-1})\)  then \(p'' \in (y_2,\dots,y_n)\) and
                                                \(p'' == p'\).
                                          \item We can now see that \(|p''|\) must be the \(MAX(|p|,|p'|)\) such that
                                                \(p''\) is an LSP \(p'' \in y\).
                                    \end{enumerate}
                              \item Thus, by induction, we may see that we have proved our proposition for Rule 3.
                        \end{enumerate}
                  \end{proof}
            \item We have now proved that each of our rules maintains our invariants, and generates
                  the desired DP table when executed for a string of any size \(n\).
      \end{enumerate}
\end{proof}

\begin{proof}[Complexity Analysis]
      \begin{enumerate}
            \item The runtime of populating the initial DP table is \(O(n^2)\).
            \item The runtime of populating the DP table \(LSP\) is sped up by memoizing
                  prior calls inside of the DP table, reducing them to table lookups which are \(O(1)\).
            \item The Worst Case situation requires we compute \(n*(n + 1)/2\) elements in the table, because it is
                  upperbounded by our invariant \(1 \leq i \leq j \leq n\).
            \item For each entry we need to then calculate recursively, the amount of work
                  is constrained to at most 4
                  applications of the prior described rules, and is followed by two
                  \(O(1)\) runtime
                  lookups in the DP table, and the subsequent maximization on those two values.
            \item Thus the runtime of computing each entry of the table is in \(O(1)\),
                  and thus with \(n^2\) entries
                  the runtime for filling our table is in \(O(n^2)\) as well.
            \item Adding these two runtimes together we get \(O(C * n^2)\) for our total runtime which is in
                  \(O(n^2)\).
      \end{enumerate}
\end{proof}


\end{document}

