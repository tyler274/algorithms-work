% preamble
\documentclass{article}
\usepackage{fullpage}
\usepackage{fancyhdr}
\setlength{\headheight}{12pt}
\usepackage[headsep=1cm,total={6.5in, 8.5in}]{geometry}

% useful packages
\usepackage{amsmath,amssymb, amsthm}

% add your own packages here
\usepackage{verbatim}

% commands for header, fill in the appropriate values
\newcommand{\codename}	{Snoop Matroidd} % type name here
\newcommand{\settitle}	{Problem Set 1} %type set title here (ex. Problem Set 3, Midterm)
\newcommand{\latesub}	{No} % type yes/no indicating late submission

% a few helpful commands 
\newcommand{\RR}{\mathbb{R}} % typing \RR prints the Blackboard R used for Real Numbers
\newcommand{\NN}{\mathbb{N}} % typing \NN prints the Blackboard N used for Natural Numbers
\newcommand{\ZZ}{\mathbb{Z}} % typing \ZZ prints the Blackboard Z used for Integers

% construct your own commands here

\newcommand{\GCD}{\textrm{GCD}}


% commands for header, don't adjust
\begin{document}
\thispagestyle{empty}
\begin{center}\framebox{\vbox{ \vspace{2mm}
                  \hbox to 6.5in {\textbf{CS~38~~~Algorithms}\hfill \textbf{\today}} \vspace{4mm}
                  \hbox to 6.5in {\Large \hfill \settitle  \hfill} \vspace{2mm}
                  \hbox to 6.5in {\textbf{Codename: } \codename \hfill \textbf{Late Submission: } \latesub} \vspace{2mm}}
      }
\end{center} \vspace*{1mm}
\pagestyle{fancy}\lhead{Codename: \codename} \chead{\settitle} \rhead{\today}

% insert solution here
\section{Problem 1}
Collaborated with: No one.

\subsection{Part A}
\begin{proof}

      \begin{enumerate}
            \item We can construct some set of functions $f$ and $g$ that are piece wise
                  functions $\forall n \in \ZZ^+$. To force the functions to be strictly increasing we will
                  multiply our input $n$
                  for all values of n by a $n^(n-1)$ or $n^n$ factor. To differentiate them, multiply
                  the alternate case by $n^n$ instead. $f(n)$ applies these to one of the two
                  cases, and $g(n)$ applies it to the other.

            \item \[ f(n) =   \begin{cases}
                              n^{n+1} & n \in Evens \\
                              n^n     & n \in Odds
                        \end{cases}
                  \]
                  \[ g(n) =   \begin{cases}
                              n^n     & n \in Evens \\
                              n^{n+1} & n \in Odds
                        \end{cases}
                  \]
            \item Each of the above functions is monotonically increasing, each increasing
                  input over the domain of $n$ always outputs a larger value as $n$ itself
                  grows. i.e. $f(n) < f(n+1) \forall n$.
            \item We will show via contradiction that $f\notin O(g)$, assume that $f \in O(g)$.
            \item For this assumption to hold, there must exist some value $p: f(n) \leq
                        p*g(n) \forall n \notin {n^*}$ where $n^*$ is some finite set $n^* \in
                        n$.
            \item If $n$ is odd than $\frac{f(n)}{g(n)} = \frac{n^{n+1}}{n^n} = n \leq p$.
            \item Contradiction! There are infinite possible values of an odd $n$ where $p
                        < n$ that dont satisfy our restrictions.
            \item $\therefore$ in the odd direction, $f() \notin O(g)$.
            \item Solving for the evens case is almost exactly the same as the even's case, we assume
                  that $g \in O(f)$, such that there is some value $p: g(n) \leq
                        p*f(n) \forall n \notin {n^*}$ where $n^*$ is some finite set $n^* \in
                        n$. We find there are infinite possible options for $p$ in this domain that
                  violate our restrictions.
            \item $\therefore$ in the even direction, $g() \notin O(f)$.
      \end{enumerate}
\end{proof}

\subsection{Part B}
\begin{proof}
      \begin{enumerate}
            \item Given that there are finite $k$ functions, there is a corresponding finite
                  set of positive integer outputs of $f(n)$ that has a determinable maximum
                  member $\forall n \in Z^+$.
            \item Since this is a finite set, we can assume there exists some function $g(n)
                        = MAX({f_i(n),\dots}) : i \in k$.
            \item $\forall n$, observe that $\forall i, f_i(n) \leq g(n)$.
            \item $\therefore f_i(n) \in O(g) \forall i$.
            \item Given that $g' : Z^+ \rightarrow Z^+ : f_i \in O(g') \forall i$, observe
                  that $f_i(n) > p_i * g'(n) \forall i$ for the finite set of values $n$.
            \item Assuming that  $p = MAX_i (p_i)$, observe that $p*g'(n) < f_i(n)$ for the
                  finite set of values $n$, and the cases where the least upper bound doesn't hold shrinks as
                  $p$ grows.
            \item Assuming that there are $u_i$ values greater than $p * g'(n) \forall
                        f_i(n)$, we can observe that $g(n) = f_i(n)$ whenever the least upper
                  bound would be
                  broken in the most egrigeous situations.
            \item $\therefore$ there are $ k * MAX_i (u_i)$ points where the least upper bound is broken at
                  most.
            \item Since the values of both parts of the bound breaking points are
                  finite, the product of the two is also finite, $\therefore g \in O(g')$.

      \end{enumerate}
\end{proof}

\subsection{Part C}
\begin{proof}

      \begin{enumerate}
            \item To prove this we will construct such a function $g(n) = MAX_{j \leq
                                    n}(f_j(n))$, observe this function is well defined over the domain of $n$.
            \item Refering to the definition of Multivariate Big O Notation (TA Notes 2.1.6),
                  observe that $f_i(n) \leq g(n) \forall i < n$.
            \item $\therefore f_i \in O(g)$ if the first value of $n$, $n_0$ is set equal to
                  $i$ by definition 2.1.6.
      \end{enumerate}
\end{proof}

\newpage
\section{Problem 2}
Collaborated with: No one.

\subsection{Algorithm}

\paragraph{\indent} We can show that there exists an algorithm for every context free grammar
in the given form that is able to determine membership in that language
in time proportional to $O(n^3)$ by constructing such an algorithm and
demonstrating its running time. This algorithm
operates by determining if a substring $u_i$ of our input string $u$
is in the given language $L$ we are checking recognition for,
and by concatenating said valid substrings we can build back up to checking
whether $w$ itself is in the language without needing to re-check our prior work. This is
a form of dynamic programming that uses a 2D table $Tab$ that has dimensions $n$ by $n$.
Entry $Tab[x,y]$ will contain every non-terminal variable that generates a substring of our input string
$u$ that begins at index $x$ and has a length of $y$, indexing from 1. If the input string
has length equal to 0 than we short circuit a solution for that.

\begin{enumerate}
      \item For $x \in 1\ldots n$:
            \subitem Look up entry $Tab[x, 1]$ and insert the set of non terminal characters ($A$)
            that have an associated rule that directs them to the terminal character $u_x$.
      \item For $y \in 2\ldots n$:
            \subitem For $x \in 1\ldots(n-y+1)$:
            \subsubitem For $z \in 1\ldots y-1$:
            \subsubitem \hspace*{6mm} For each variable $B \in Tab[x, z]$ and $C \in
                  Tab[x+z, y-z]$, locate all rules of the form $A \rightarrow BC$ and append
            the variable $A$ to the set of values in the DP table at $Tab[x,z]$.
      \item If the input string's length is 0, such that $u=\epsilon$, see if $S \rightarrow
                  \epsilon \in R$ and return True if and only if it is in the set $R$, and
            return False if it is not.
      \item If the input string's length is not 0 such that $u \neq \epsilon$ then if $S \in
                  Tab[1, n]$ return True. Return False if its not.
\end{enumerate}

\subsection{Proof of Complexity}

\begin{proof}:

      \begin{enumerate}

            \item
                  The first loop initialized the DP table $Tab$ and runs $n$ times in the length of the string.
            \item
                  There is a constant number of
                  non-terminals rules in $R$ of the (i. $A\rightarrow BC$) form.
            \item Thus the initial loop runs in O(n).
            \item
                  There are three nested loops that each run in the worst case $n$ times
                  as well.
            \item
                  The innermost loop's work runs in constant time over the domain of the
                  variables.
            \item Steps 2 of the algorithm thus run in $O(n^3)$.
            \item Steps 3-4 of the algorithm each run in constant time.
            \item
                  We thus have an algorithm that determines membership in any language with the form of $L$
                  in $O(n + n^3 + c)$ time which can be simplified to $O(n^3)$.
      \end{enumerate}
\end{proof}

\subsection{Proof of Correctness}
\begin{proof}
      This is a parse tree that proves the base case (substring) and then
      builds up the wider parsing tree off our base case derived from the set of Rules
      provided by the CFL. By Induction we will see
      that the algorithm correctly determines recognition by the language.
      \begin{enumerate}
            \item Base Case: (sub)strings of length $y=1$.
                  \subitem The (sub)string has to be the product of one of the rules $\in R$,
                  such that the substring $u_i$ is the terminal of one of the rules with the
                  form of $A\rightarrow u_i$.
                  \subitem Thus by construction, so long as the Alphabet remains the same as
                  the CFL for all $u$, we can observe that the values we insert into the row
                  $y=1$ is valid for the CFL.
            \item Induction step: for substrings of lengths $y>1$
                  \subitem Each step of the algorithm only depends on rows that have already
                  been filled in, starting at the base case that we proved is correct.
                  \begin{enumerate}
                        \item Each substring $v$ of the stated length needs to have been
                              generated by a rule $\in R$ that has the form $A\rightarrow BC$,
                              in which sub strings B and generate the component substrings of $b
                                    = b_1 b_2$.
                        \item In the middle loop of our algorithm (For $z \in 1\ldots y-1$), the
                              algo generates entries for every way we could split the sub string $v$
                              into its two component pieces.
                        \item By definition the substrings must
                              have a length smaller than $y$, and the innermost loop of the
                              algorithm will find every rule that could generate the substring
                              and enter it into $Tab$.
                        \item Based on our induction, each row in the table $y$ is correct, as
                              it will contain the substring of the input $u$ begging at index
                              $x$ and possessing length $y$.
                  \end{enumerate}
            \item The case of length equal to 0, or the empty string $\epsilon$ is handled by
                  the definition of the language.
            \item Using the above proof of correctness on induction that the table has
                  been generated correctly, we know that the item in the table at $Tab[n,1]$
                  contains the set of all possible variables that generate the string itself.
                  If the starting symbol $S$ is in the set found in this entry in the table,
                  then the input string $u$ is one that is produced by the CFG, and the
                  algorithm should return True.
            \item If the starting symbol $S$ is not in the set of variables in $Tab[n,1]$,
                  then the input string $u$ must not be in the language, as that set contains all
                  valid variables that generate the string. Thus the algorithm will return
                  False in that case.
      \end{enumerate}

\end{proof}

\newpage
\section{Problem 3}
Collaborated with: no-collab

\subsection{Algorithm}
\paragraph{\indent}
We will construct a DP table $Tab[u,v]$ with dimensions $(m + 1)$ times $(n + 1)$.Each
entry in the table stores the length of the SCS of the strings $a_u =
      (a_1,a_2,\ldots,a_u)$ and $b_v = (b_1,b_2,\ldots,b_v)$ (corresponding to the input strings
x and y). The variable $Z$ is string that will
represent the current instance of the SCS for input strings $x$ and $y$.
Call this function $SCS(x,y)$.
\begin{enumerate}
      \item For $u \in 0\ldots m$:
            \begin{enumerate}
                  \item For $v \in 0\ldots n$:
                        \begin{enumerate}
                              \item If $u=0$, insert $v$ into the DP table at $Tab[u,v]$.
                              \item Else If $v=0$, insert $u$ into the DP table at $Tab[u,v]$.
                              \item Else If $a_u$[u] = $b_v$[v], insert $Tab[u-1, v-1] + 1$
                                    into the DP table at $Tab[u,v]$.
                              \item Else, insert $($MIN$(Tab[u-1,v], Tab[u, v-1]) + 1)$ into the
                                    DP table at $Tab[u,v]$. Where MIN() is a function that
                                    returns the minimum of its two input values.
                        \end{enumerate}
            \end{enumerate}
      \item Set $Z$ to the empty string $\epsilon$.
      \item Set $u$ to $m$.
      \item Set $v$ to $n$.
      \item Loop the following while $u>0$ \&\& $v>0$.
            \begin{enumerate}
                  \item If $x[u]$ == $y[v]$, insert $x[u]$ into the front of the string
                        $Z$, decrement $u$ by 1, and decrement $v$ by 1.
                  \item Else If SCS$(a_u, b_{v-1}) < $SCS$(a_{u-1}, b_v)$, insert $y[v]$
                        into the front of $Z$, and decrement $v$ by 1.
                  \item Else, Insert $x[i]$ into the front of $Z$, and decrement $u$ by 1.
            \end{enumerate}
      \item If $u = 0$, then insert $x[0..u]$ into the front of string $Z$.
      \item If $v = 0$, then insert $y[0..v]$ into the front of string $Z$.
      \item The SCS is now stored in variable $Z$, and its length in the DP table entry at
            $Tab[m,n]$.
\end{enumerate}

\subsection{Proof of Complexity}
\begin{proof}
      \begin{enumerate}
            \item The DP table $Tab$ is populated in $O(m*n)$ iterations each of which
                  runs in constant time.
            \item The backtracing calculation of the SCS using this has a worst case time
                  complexity of $O(m+n)$, in the case where both strings must be iterated
                  through.
            \item The runtime complexity is thus dominated by the $O(m*n)$ component,
                  which thus represents the overall runtime complexity.
      \end{enumerate}


\end{proof}

\subsection{Proof of Correctness}
\begin{proof}
      We will prove that this algorithm is correct using induction, more specifically that
      each entry in our DP table has the length of the SCS of the substrings of the two
      input strings.
      \begin{enumerate}
            \item Base Case(s): $u = 0$ OR $v = 0$
                  \begin{enumerate}
                        \item If either string is empty, then the SCS is trivially the
                              other input string, and thus the correct length of the SCS
                              is inserted.
                  \end{enumerate}
            \item Inductive Step:
                  \begin{enumerate}
                        \item For substrings with length $u,v \in \ZZ^+$, we may make the
                              inductive assumption that the table has been correctly
                              populated with the SCS lengths of the substring pairs, as
                              they would have been filled in during the processing of
                              strings $a_u$ and $b_v$:
                              \begin{enumerate}
                                    \item $(a_{u-1}, b_{v-1})$
                                    \item $(a_u,b_{v-1})$
                                    \item $(a_{u-1},b_v)$
                              \end{enumerate}
                        \item In the case of both strings ending in the same character we can reduce the problem of finding the SCS of the sub
                              strings $a$ and $b$ by observing that the SCS of the two
                              strings must contain the final letter of both strings if the two strings both end in that leter, and that the
                              smallest common supersequence must have that character as
                              the final character in the sequence. The SCS is thus finding
                              the SCS of the parts of the (sub)strings before that final
                              common character. If our inductive assumption holds true as
                              we assume, then the size of the SCS for the DP table entry
                              at $Tab[u-1,v-1]$ would have been correctly set, and we thus
                              derive the correct length and content of the SCS.
                        \item In the case that the last character of each (sub)string are
                              not the same, we can observe that the SCS must contain the last
                              character of one of the input strings as the SCS' last
                              character. The SCS string before that character will be the
                              other (sub)string. Via induction the SCS for both possible
                              branches has already been set, and we can thus derive the
                              correct value of the SCS and its length to populate the
                              entry in the DP Table.
                        \item Thus every substring case has been populated correctly in
                              the DP table, and the correct length of the SCS has been set in
                              $Tab[m,n]$.
                        \item We can observe that the backtracing algorithm for the SCS
                              generates the correct sequence by noting that the insertion of
                              characters into the front of the string $Z$ generates a super
                              seuqence as it includes each member of both of the inputs string $x$ and
                              string $y$.
                        \item As $Z$ is generated by following the table it's
                              length is guranteed to match the value found in the DP table
                              entry at $Tab[m,n]$, which we already proved will be correct
                              previously.
                  \end{enumerate}
      \end{enumerate}


\end{proof}

\end{document}

