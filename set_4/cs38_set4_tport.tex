% preamble
\documentclass{article}
\usepackage{fullpage}
\usepackage{fancyhdr}
\setlength{\headheight}{12pt}
\usepackage[headsep=1cm,total={6.5in, 8.5in}]{geometry}

% useful packages
\usepackage{amsmath,amssymb, amsthm}

% add your own packages here
\usepackage{verbatim}
% \usepackage[style=verbose]{biblatex}
\usepackage{enumitem}
% \usepackage{filecontents}
% \begin{filecontents}{myreferences.bib}
%     @online{foo12,
%       year = {2012},
%       title = {footnote-reference-using-european-system},
%       url = {http://tex.stackexchange.com/questions/69716/footnote-reference-using-european-system},
%     }
% \end{filecontents}

% % Save the bib to disk (or update it).
% \addbibresource{myreferences.bib}

% commands for header, fill in the appropriate values
\newcommand{\codename}	{Snoop Matroidd} % type name here
\newcommand{\settitle}	{Problem Set 4} %type set title here (ex. Problem Set 3, Midterm)
\newcommand{\latesub}	{No} % type yes/no indicating late submission

% a few helpful commands 
\newcommand{\RR}{\mathbb{R}} % typing \RR prints the Blackboard R used for Real Numbers
\newcommand{\NN}{\mathbb{N}} % typing \NN prints the Blackboard N used for Natural Numbers
\newcommand{\ZZ}{\mathbb{Z}} % typing \ZZ prints the Blackboard Z used for Integers

% construct your own commands here

\newcommand{\GCD}{\textrm{GCD}}

% commands for header, don't adjust
\begin{document}
\thispagestyle{empty}
\begin{center}
      \framebox[6.5in]{
            \vbox{
                  \vspace{2mm}
                  \makebox[6.3in][l]{\textbf{CS~38~~~Algorithms}\hfill \textbf{\today}}
                  \vspace{4mm}
                  \makebox[6.5in][c] {\Large \hfill \settitle{\hfill}}
                  \vspace{2mm}
                  \makebox[6.3in][l] {\textbf{Codename: } \codename{ \hfill \textbf{Late Submission:}
                              \latesub}}
                  \vspace{2mm}
            }
      }
\end{center}
\vspace*{1mm}
\pagestyle{fancy}\lhead{Codename: \codename} \chead{\settitle}
\rhead{\today}

% insert solution here
\section{Problem 1 (6 points)}
Collaborated with: No One

\subsection{Proposition}
\paragraph{}
We have seen that the Activity Selection problem can be solved with a greedy algorithm
thanks to a certain \textit{exchange} property. This exchange property, however, is not
precisely the same as the one we encountered in matroid theory.

A natural suggestion is to check whether the family of all non-overlapping sets of
activities, forms a matroid. Show it is a hereditary family but does not satisfy the
Steinitz exchange axiom.

\subsection{Proof}
\begin{proof}
      \begin{enumerate}
            \item The Activity Selection problem takes as input a set of activities, and
                  returns a solution to the problem as a subset of those activities.
            \item Each activity is denoted by a unique ID, and has a start time \(t_s\),
                  and a time to be finished by \(t_f\).
            \item For a solution to be valid the result must be a non overlapping subset
                  of activities.
            \item Assume that this can be represented and solved using a matroid solution.
            \item Consider the three activities \({A_1, A_2, and A_3}\) where \(A_1\)
                  overlaps with \(A_2\) and with \(A_3\), but neither \(A_2\) or \(A_3\)
                  overlap with one another.
            \item There are only two valid solutions, the subset consisting of \(A_1\),
                  and the subset consisting of \({A_2, A_3}\).
            \item Contradiction, the Steinitz Exchange axiom is not satisfied as \(A_2\)
                  or \(A_3\) can't extend our solution of \({A_1}\).
            \item Thus, the exchange axiom is not satisfied for the Activity Selection
                  problem.
            \item To prove the Hereditary Axiom, assume that for the sake of contradiction
                  if \(J \in F\) and \(I \subseteq J\), then \(I \notin F\).
            \item Given our above set of three activities, observe that any individual element of
                  \({A_1, A_2, A_3}\) must me a subset of the whole, and that any equal
                  subset must by extension be an element of \(F\).
            \item Contradiction, as we assumed that \(I \subseteq J\) implies \(I \notin
                  F\), and this violated our definition of set theory.
            \item Thus, the Hereditary Axiom holds.
      \end{enumerate}
\end{proof}

\newpage
\section{Problem 2 (6 points)}
Collaborated with: No one

\subsection{Proposition}

\paragraph{\indent}
Let \(G = (V, E, w)\) be a weighted, undirected graph. For a spanning tree \(T\) of \(G\),
let \(L(T)\) be the sorted list of edge weights of \(T\) (with repetition). Show that for
any two minimum spanning trees \(T_1, T_2, L(T_1) = L(T_2)\).

\subsection{Proof}
\begin{proof}
      \begin{enumerate}
            \item We will prove this using Kruskal's algorithm as described in lecture 9\footnote{Lec09H.pdf}.
            \item Assume that \(L(T_1) \neq L(T_2)\) for sake of contradiction.
            \item Let \(L_d = L(T_1) \ominus L(T_2) \) be the disjunctive union of the
                  sets.
            \item Choose an edge in only one of the trees \( e \in T_1 \ominus T_2\) such
                  that the weight \(w(e) = \min(L_d) \).
            \item Without loss of generality, there must always be an edge that satisfies the above, as not every edge
                  with a minimized weight can exist in both of the trees, or \(\min(L_d)
                  \notin L_d \).
            \item Let \( e\in T_1 \) and assume that \(T_1\)
                  has more edges with weight of \(\min(L_d)\) than in \(T_2\).
            \item Using Kruskal's algorithm for cutting, find all the edges of \(T_2\)
                  that cross the cut of \(T_1\) on \(e\).
            \item If we find an edge \(f\) such that \(w(e) = w(f)\) then replace \(e\)
                  with \(f\).
            \item We can observe via inspection that the modified \(T_1\) remains a minimal spanning
                  tree, and has the same sorted list of edge weights as our starting
                  \(T_1\).
            \item Iterating on the above, we remove two elements from \(L_d\)
                  and one edge from the set of possible \(e\) at each iteration.
            \item As the set universe we are working with is finite, there must be finite
                  iterations to get to the point where the all the edges in the
                  intersection of \(T_2\) and the cut of the new \(T_1\) on each \(e\) has
                  a weight distinct from \(w(e)\).
            \item Thus we can choose an edge \(f \in \) the intersection of the cut on
                  \(T_1\) and \(T_2\) where we can replace \(e\)and \(f\) to create
                  another spanning tree \(T' = \{^{(T_1 \backslash \{e\}) \cup \{f\}, w(f)
                              < w(e)}_{(T_2 \backslash \{f\}) \cup \{e\}} \).
            \item The MST \(T'\) has a total edge weight less than either \(T_1\) or
                  \(T_2\), which is a contradiction as we were given that \(T_1\) and
                  \(T_2\) are MSTs.
            \item Thus we can conclude that \(L(T_1) = L(T_2)\) for any two minimum
                  spanning trees of some graph \(G\).
      \end{enumerate}
\end{proof}

\newpage
\section{Problem 3 (6 points)}
Collaborated with: No collab

\subsection{Proposition}

\paragraph{}
As you probably remember, the graph coloring problem --- given an undirected graph, color
the vertices with the fewest possible colors so that every edge is bi-chromatic --- is
NP-hard. For graphs with extra structure, the problem may be easier, and this problem is
concerned with one such situation.

An \textit{interval graph} is one in which each vertex
\(v\) is labeled with an interval on the real line \(s_v, t_v\), \(s_v < t_v\), and there
is an edge \( (u,v) \) if the two intervals \(s_u, t_u\), \(s_v, t_v\) intersect
internally.

Design an \(O(n \log n)\)-time algorithm to optimally color an interval graph having \(n\)
vertices. You may assume that basic arithmetic on, or comparisons between, interval
endpoints (the \(s\)'s and \(t\)'s), can be performed in unit time.


\subsection{Algorithm}
\begin{enumerate}
      \item We will devise a greedy algorithm using the Activity Selection method given in
            the TA Notes\footnote{TA Notes 5.2.1, Lec10H.pdf} that optimally colors an interval graph of
            \(n\) vertices.
      \item Label our integer colors \( c \in \ZZ \) with real intervals \( (s(c), t(c))
            \in \RR \) and initialize \(c := 0\) to store the largest color index that we
            use.
      \item Let \(Q\) be a queue containing each color thats been utililized, and
            initialize it to the empty set \( Q := \emptyset \).
      \item Let \(P = \text{MakePriorityQueue} (P) \) \( \{ s_i, t_i : 1 \leq i \leq n \}
            \) be a priority queue as described in lecture 10\footnote{Lec10H.pdf} and the
            TA Notes\footnote{TA Notes 5.8.1}. \(P\) is filled with every
            interval end \(t_v\) along with its interval and we tie break for some pair
            \( (t_u,t_v) \) preferentially on the \(t_u, t_v\) values (interval ends). If
            for some pair \( (u,v) \) the values of \( t_u = t_v \) then tie break on the
            \( s_u, s_v\) values (interval beginnings).
      \item While \(P \neq \emptyset \) keep looping executing the following
            \begin{enumerate}
                  \item Let \(a := \text{DeleteMin}(P)\).
                  \item If \(a \in s\), then if \(Q \neq \emptyset \) then assign color
                        \(b := Pop(Q)\) else assign \(c := c + 1\) and \(b := c\), then
                        assign \(s(b) := s_a\) and \(t(b) := t_a\) where \(s(b)\) and
                        \(t(b)\) are the reals labeling color \(b\)'s interval.
                  \item Else if \( a \notin s \), then \( s(b) := s(a)\) and \(t(b) =
                        t(a)\), then \(\text{Push} (Q, b) \).
            \end{enumerate}
      \item When we have finished the above loop iterations and have empty \(P\) we will then have an optimal
            coloring of our interval graph in the variable \(c\).
\end{enumerate}

\subsection{Proof of Correctness}
\begin{proof}
      \begin{enumerate}
            \item Let \(C\) be the supposed maximum number of colors used during the
                  running of the algorithm, and thus the value of \(c\) after running our
                  algorithm to completion (for some labeled interval graph \(G = (V,E)\)).
            \item When the color \(C\) is assigned to a vertex (end interval), each color
                  from the color at \(1\) to the color at \(C-1\) is assigned to some
                  vertex \( \in V\).
            \item This algorithm only allows a vertex to be specific color if and only if it maps to
                  that color's interval.
            \item Thus the set of vertices we are given contains \(C\) mutually
                  overlapping colors.
            \item Thus \(C\) is a lower bound on the numbers of colors needed to optimally
                  color the graph.
            \item As the algorithm is designed to use exactly \(C\) colors it already uses
                  the lower bound of the number of colors necessary.
            \item Thus our algorithm performs a minimal coloring of the graph as desired.
      \end{enumerate}
\end{proof}

\subsection{Proof of Complexity}
\begin{proof}
      \begin{enumerate}
            \item This algorithm runs in \(O (n \log n) \) as we can observe that there
                  are \(2n\) ends of the interval that populate our priority queue, and
                  thus \(2n\) bounds the number of iterations.
            \item For each of these interval ends we do an \( O(\log n ) \) Insert and an
                  \(O(\log n)\) DeleteMin as described in the notes from lecture 10\footnote{Lec10H.pdf}.
            \item Thus we have an algorithm that runs in \( O((2n) (2 \log n)) = O(n \log
                  n) \).
      \end{enumerate}
\end{proof}

\end{document}

